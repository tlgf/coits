%  title: CANopen in the Shell
%   date: 18-09-2017
% author: T Flynn

\documentclass{beamer}
\usepackage{graphicx}
\usepackage{xpatch}
\usepackage{xcolor}
\usepackage{amsthm}

% From https://tex.stackexchange.com/questions/11954/
\newcommand {\framedgraphic}[2] {
    \begin{frame}{#1}
        \begin{center}
            \includegraphics[width=\textwidth,height=0.8\textheight,keepaspectratio]{#2}
        \end{center}
    \end{frame}
}

\setbeamertemplate{blocks}[rounded][shadow=false]
\setbeamercolor*{block title}{fg=blue!100,
  bg= blue!10}
\setbeamercolor*{block body}{fg= blue,
bg= blue!5}

\makeatletter
\patchcmd\beamer@@tmpl@frametitle{\insertframetitle}{\insertsection$ $- \insertframetitle}{}{}
\makeatother

\title{CANopen in the Shell}
\subtitle{Managing and Developing CANopen Applications on Linux}
\author{Thomas Flynn}

\usetheme{AnnArbor}
\usecolortheme{dolphin}

\mode<presentation>{
\begin{document} 

% title page
  \begin{frame}
    \framesubtitle{\textbf{S}ydney \textbf{L}inux \textbf{U}ser \textbf{Group}}
    
    \titlepage
  \end{frame}

  
\section*{Introduction}

% informal intro  
\subsection*{About me}
\begin{frame}
    \huge{Me (Tom Flynn):}
    \begin{itemize}
      \item{BEng. Mechatronics}
      \item{Linux User}
      \item{Industrial Electronics}
      \item{Engineer}
    \end{itemize}
    \end{frame}
  
% formal intro
\subsection*{Overview}
  \begin{frame}{Outline}
    \begin{enumerate}
    \item{Controller Area Networks - What, Why, How?}
      \pause
    \item{Unleash the Protocols}
      \pause
    \item{CAN and the Kernel}
      \pause
    \item{The CAN FOSS Tool Kit}
      \pause
      \item{Smooth and Unproblematic Demonstration}
   \end{enumerate}        
\end{frame}
  
  % funny
  \framedgraphic{GNU Image Manipulation}{./images/title_slide}



  \section{Controller Area Networks - What, Why, How?}
    \begin{frame}
      \begin{Huge}{A Timeline:}\end{Huge}
      \begin{itemize}
      \item{\textbf{1980s} Continuing increase of electronic sensors in vehicles. Signalling limited by environmental noise, complexity of wiring, limited I/O on controllers.}
        \pause
        \item{\textbf{1983} Development of the CAN bus concept at Robert Bosch GmbH}
          \pause
        \item{\textbf{1986} CAN bus 'protocol' released at SAE conference.}
          \pause
        \item{\textbf{1987} First CAN transceiver chips released by Intel and Phillips}
          \pause
        \item{\textbf{1988} First Production vehicle to feature CAN-based system.}
          \pause
        \item{\textbf{1993} ISO CAN standard released, ISO 11898}
          \pause
        \item{\textbf{1995} CAN in Automation (CiA) release CiA 301, CANopen application layer and communication profile -  $Members Only$.}
          \pause
        \item{\textbf{2011} CiA 301 V 4.2 made public - $Open$}
      \end{itemize}
  \end{frame}

% Empty Template
    
    \begin{frame}  
      \begin{block}{What is a Controller Area Network?}
        A \textcolor{green}{network} of \textcolor{red}{nodes} \textcolor{magenta}{exchanging} \textcolor{orange}{messages}.
      \end{block}      
    \end{frame}

    \framedgraphic{Network}{./images/generic_network}

    \framedgraphic{Exchange}{./images/can_signaling}

    \framedgraphic{Messages}{./images/CAN-Bus-frame_in_base_format_without_stuffbits}
%% % Empty Template
%% \begin{frame}
%%     \frametitle{}
%%     \framesubtitle{}
%%     %content goes here
%%   \end{frame}


}
\end{document}

