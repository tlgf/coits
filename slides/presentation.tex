%  title: CANopen in the Shell
%   date: 18-09-2017
% author: T Flynn

\documentclass{beamer}
\usepackage{graphicx}

\title{CANopen in the Shell}
\subtitle{Managing and Developing CANopen Applications on Linux}
\author{Thomas Flynn}

\usetheme{AnnArbor}
\usecolortheme{dolphin}

\mode<presentation>{
\begin{document} 

% title page
  \begin{frame}
    \framesubtitle{\textbf{S}ydney \textbf{L}inux \textbf{U}ser \textbf{Group}}
    
    \titlepage
  \end{frame}

  
\section{Introduction}

% informal intro  
\subsection{About me}
\begin{frame}
    \frametitle{Introduction}
    \huge{About Me (Tom Flynn):}
    \begin{itemize}
      \item{BEng. Mechatronics}
      \item{Linux User}
      \item{Industrial Electronics}
      \item{Engineer}
    \end{itemize}
    \end{frame}
  
% formal intro  
  \begin{frame}{Outline}
    \begin{enumerate}
    \item{Controller Area Networks - What, Why, How?}
      \pause
    \item{Unleash the Protocols}
      \pause
    \item{CAN and the Kernel}
      \pause
    \item{The CAN FOSS Tool Kit}
      \pause
      \item{Smooth and Unproblematic Demonstration}
   \end{enumerate}        
\end{frame}

  % funny
    \begin{frame}
    \frametitle{GNU Image Manipulation}
    \begin{center}
    \includegraphics[width=7.5cm,height=7.5cm, keepaspectratio]{./images/title_slide}  
    \end{center}
   
    \end{frame}


    
  
% Empty Template
\begin{frame}
    \frametitle{}
    \framesubtitle{}
    %content goes here
  \end{frame}


}
\end{document}
